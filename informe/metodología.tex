\section{Objetivo}

El objetivo de este trabajo se sentrará en explorar diversas tecnicas para sintetizar habla en español con prosodia extranjera. 


Como primer paso se utilizará festival y festvox para realizar alineamiento forazado[1] sobre el corpus de audios y sus transcripciónes grafemicas para así obtener transcripciones foneticas. Luego se procederá a utilizar los audios junto con la transcripción fonetica para realizar el entrenamiento utilizando HTS.

Contaremos con tres? dos? corpus de datos distintas para el entrenamiento, dos en castellano y una en ingles. Luego del entrenamiento y con los modelos foneticos ya generados utilizaremos hts_engine para interpolar entre ellos y lograr nuevos modelos que mezclen los features acusticos con distintos grados de ingles y de castellano.

Un desafío que se presenta para este trabajo es el mapeo de los fonemas del ingles al castellano. Para empezar, la transcripcion fonetica realizada por festival de las oraciones en ingles puede utilizar 50 simbolos distintos, mientras que la transcripción fonetica del castellano utiliza 31. Habiendo ademas muchos simbolos sin equivalencia. (por ejemplo, con el fonema /rr).

Para resolver esto desarollamos una solución adhoc que consist en desarrollar una función sobreyectiva que permita tener cubiertos los 31 fonemas del castellano por alguno del ingles.

%[1] se usa EHMM alignment


\section{Metodología}

Para este trabajo se utilizará HTS para el entrenamiento de los diversos modelos entrenados.
