El Speaker-Adaptative Training es una técnica que permite tomar un modelo ya entrenado y adaptarlo para asimilar características de un nuevo hablante. Esta técnica nace de la idea de que construir un corpus de datos es costoso tanto en espacio de almacenamiento, tiempo de grabación y etiquetado, por lo que resulta mas económico generar una nueva voz sintética a partir de un modelo bien generado y adaptándolo luego con características del nuevo corpus.

Nuestro objetivo para este trabajo es utilizar esta herramienta para aproximar las características de uno de los hablantes al otro para que sus identidades fueran indistinguibles.

Como prueba de concepto se tomó el corpus \textit{CMU-ARCTIC-SLT} y se le realizó Speaker Adaptation junto con \textit{loc1\_pal} utilizando la demo presente en la sección de descargas de HTS para el entrenamiento. 

Dentro del adaptative training existen varias técnicas, en este trabajo utilizaremos offline supervised adaptation, que tiene como requisito adicional conocer los oraciones del segundo corpus. 

Las pruebas no resultaron concluyentes, las oraciones sintetizadas no solamente perdían la identidad del hablante original sino también sus características fonéticas. Dicho de otra manera, si lo que buscábamos era que obtener un hablante ingles que pudiera ser reconocido como el mismo locutor que \textit{loc1\_pal} pero con sus características fonéticas intactas (una pronunciación suavizada del fonema /\textipa{r}/ (\textit{perro}), por ejemplo), lo que en realidad obtuvimos fue una voz idéntica a \textit{loc1\_pal}. Si bien existen indicios que indican que es posible generar un modelo con las características deseadas\cite{statisticalParam} \cite{speakerSim}, dada la complejidad del método y los largos periodos que son necesarios para entrenar un modelo ($36$ horas aproximadamente) decidimos abandonar este camino y continuar con la fase de evaluación perceptual sobre las voces generadas con las técnicas de interpolación.