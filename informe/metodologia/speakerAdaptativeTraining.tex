El Speaker-Adaptative Training técnica permite tomar un modelo ya entrenado y adaptarlo para asimilar características de un nuevo hablante. Esta técnica nace de la idea que construir un corpus de datos es costoso tanto en espacio de almacenamiento, tiempo de grabación y etiquetado, por lo que resulta mas económico generar una nueva voz sintética a partir de un modelo bien generado y adaptándolo luego con características del nuevo corpus.

Nuestro objetivo para este trabajo es utilizar esta herramienta para aproximar las características de uno de los hablantes al otro para que sus identidades fueran indistinguibles.

Como prueba de concepto se tomó el corpus \textit{CMU-ARCTIC-SLT} y se le realizó Speaker Adaptation junto con \textit{loc1\_pal} utilizando la demo presente en la sección de descargas de HTS para el entrenamiento. Las pruebas no son concluyentes ya que las oraciones sintetizadas no solamente pierden la identidad del hablante original sino también sus características fonéticas. Si bien existen indicios que indican que es posible generar un modelo con las características deseadas, dada la complejidad del método y los largos periodos que son necesarios para entrenar un modelo ($36$ horas aproximadamente) decidimos abandonar este camino y continuar con la fase de experimentación.