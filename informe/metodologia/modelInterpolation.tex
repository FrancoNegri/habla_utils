Una vez generados ambos modelos con el mismo repertorio fonético procedemos a experimentar y evaluar de manera informal la efectividad del método. Para ello tomamos el modelo generado con \textit{loc1\_pal} y lo interpolamos con \textit{CMU-ARCTIC-SLT} con diferentes pesos entre ellos. Esta interpolación, a cargo de HTS\_engine, consiste a grandes rasgos en tomar ambos HMMs e interpolar sus características fonéticas y prosódicas para obtener un nuevo modelo. Para una explicación más detallada del tema ver la Sección \ref{interpolationTeory}

Para grados cercanos al $90\%$ de castellano + $10\%$ de inglés obtenemos los resultados esperables: las oraciones sintetizadas tienen un marcado acento castellano. Asimismo, en el otro extremo, $10\%$ de castellano + $90\%$ de inglés, la voz sintetizada, al igual que lo que se describió en la sección anterior, presenta problemas fonéticos graves, haciendo que las oraciones resulten poco naturales y difíciles o imposibles de comprender. Para grados intermedios de interpolación, $70\%$ de castellano + $30\%$ de inglés y $40\%$ de castellano + $60\%$ de inglés, observamos resultados más cercanos a los esperados, pudiendo apreciar en las oraciones sintetizadas detalles distintivos como el fono [\textipa{R}] más suavizado, o la pronunciación de vocales más abiertas, pero aún conservando cierto grado de inteligibilidad. De esta misma manera también pudimos apreciar en las oraciones sintetizadas cierta prosodia no familiar que también podría ser adjudicada a un hablante extranjero.  

Como un efecto colateral de la interpolación, pudimos apreciar que cuanto más cercano es el grado de interpolación al modelo entrenado con \textit{loc1\_pal}, las características fonéticas más se asemejan a las de las oraciones del corpus \textit{loc1\_pal}. De manera análoga, cuanto más porcentaje de inglés tiene la mezcla, la voz más se asemeja a \textit{CMU-ARCTIC-SLT}. Si bien esto no es un defecto importante, con el motivo de cambiar el menor número de variables en la experimentación sería deseable que la voz no presentara distintas características para distintos grados de interpolación.

Como una posible solución surge la posibilidad de utilizar Speaker-Adaptive Training sobre uno de los modelos, que describiremos a continuación.