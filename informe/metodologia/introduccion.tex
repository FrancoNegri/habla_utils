
\noindent Un sistema de Text To Speech (TTS) es aquel que genera habla artificial a partir de un texto de entrada. En la actualidad estos sistemas se encuentran incluidos en muchas aplicaciones domesticas, como por ejemplo navegación por GPS, asistentes personales inteligentes (como es el caso de SIRI), ayuda para personas no videntes, traducción automática, etc.

\noindent En las últimas décadas se han visto grandes progresos en este campo, siendo capaces de modelar con cierto grado de efectividad cuestiones tales como la prosodia del hablante, sus emociones, etc. Si bien actualmente se considera que el estado del arte para la síntesis es el entrenamiento con redes neuronales profundas (DNN), una técnica todavía utilizada es la que utiliza Modelos Ocultos de Markov mas modelos de Mezcla de Gaussianas (HMM+GMM) que, a partir de un corpus de datos de entrenamiento, extrae información acústica y genera un modelo probabilístico que permita sintetizar habla. 

\noindent Consideramos que este método si bien no nos permitirá obtener la mejor voz posible, nos permitirá entender mejor los features con los que trabajamos y por lo tanto modificarlos según consideremos conveniente, algo que puede volverse mas complicado cuando se esta trabajando con redes neuronales profundas.

\noindent En este trabajo de tesis se estudia una manera posible de generar un TTS basado en HMMs capaz de sintetizar habla en español con acento extranjero. Las razones por las que podría querer diseñarse un sistema con estas características varían desde un punto de vista puramente técnico, ya que un sistema así permitiría la utilización de corpus de entrenamiento de hablantes no nativos para la generación de una nueva voz, pasando por cuestiones lingüísticas, como es poder vislumbrar el limite en que un acento deja de parecernos local para pasar a ser extranjero y cuestiones psicológicas: lograr distintos efectos sobre el usuario, quien podría reaccionar distinto ante diferentes acentos.

\noindent En el transcurso de este trabajo se espera además evaluar la prosodia y la fonética del modelo generado con estas características, como así también evaluar su inteligibilidad. Además pretenderemos evaluar la efectividad de técnicas de speaker adaptation cuando se utilizan corpus de distintas nacionalidades con repertorios fonéticos disimiles (para este caso de estudio: castellano e inglés)

\noindent Para este trabajo nos basaremos fuertemente en la síntesis/análisis mel-cepstral, speech parameter modeling usando HMMs y speech parameter generation usando HMMs, como es descripto en la disertación doctoral \textit{Simultaneous Modeling of phonetic and prosodic parameters, and characteristic conversion for hmm-based text-to-speech systems} del Profesor Tadashi Kitamura, Nagoya Institute of Technology\cite{phoneticAndProsodic}.


Como introducción a este trabajo comenzaremos analizando las decisiones de metodología utilizadas a lo largo de la investigación, así también como detalles teóricos como el mapeo de fonemas necesario para adaptar el repertorio fonético del inglés al castellano, etc.

En las siguiente secciones se detallarán distintos temas que fueron necesarios abordar para llevar a cabo esta tesis. En orden de aparición estos son:

\begin{enumerate}
\item Realizar un etiquetado fonético de distintos corpus de audios.

\item Realizar un mapeo entre los fonos del castellano y los del inglés. Estos serán necesarios en el paso siguiente donde será requerimiento indispensable tener modelados los mismos fonos para todos los modelos.

\item Realizar el entrenamiento de los HMM+GMM. Para esto contaremos con el framework de modelado de HMMs HTS. 

\item Utilizar las herramientas provistas por HTS para interpolar entre modelos y poder sintetizar habla con distintos grados de fonética y prosodia inglesa.

\end{enumerate}

Para finalizar el apartado teórico discutiremos algunos aspectos implementativos de HTS y las otras herramientas utilizadas en este trabajo.
