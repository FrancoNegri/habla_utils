Un sistema de \textit{Text To Speech} (TTS) es aquel que genera habla artificial a partir de un texto de entrada. En la actualidad estos sistemas se encuentran incluidos en diversas aplicaciones domesticas: navegación por GPS, asistentes personales inteligentes (como es el caso de SIRI), ayuda para personas no videntes, y traducción automática, entre otros.


En las últimas décadas se han visto grandes progresos en este campo. Algunos sistemas son capaces de modelar con cierto grado de efectividad cuestiones tales como el acento, el tono y la entonación de un hablante (es decir, su \textit{prosodia}). Por otro lado tambien se ha visto un progreso en modelar emociones, o cuestiones prosodicas que pueden darnos a entender que una oración es una pregunta, una orden, etc \cite{prosodiaYEntonacion} \cite{prosodiaYEntonacion2}. 


En este trabajo de tesis se estudia una manera posible de generar un TTS basado en HMMs capaz de sintetizar habla en español con acento extranjero. Existen muchos motivos por los que podría construirse un sistema con estas características. Por ejemplo en investigaciones de carácter lingüístico, podría querer utilizarse para vislumbrar limites en que un acento deja de parecernos local para pasar a sonar extranjero. Por otro lado en investigaciones de carácter psicológico, podría querer utilizarse para medir la confianza que deposita un oyente sobre hablantes de distinta nacionalidad: por poner un ejemplo, al pedir indicaciones de como llegar a algún lugar uno no deposita el mismo nivel de confianza si la persona que responde suena oriundo de la localidad que a alguien que suena extranjero. Por ultimo un sistema así podría quererse construir por temas puramente técnicos ya que permitiría generar una voz en castellano, por ejemplo, combinando una voz previamente construida en algún otro idioma y un pequeño corpus de datos en castellano.


Actualmente se considera que el estado del arte para la síntesis de voz es el entrenamiento con redes neuronales profundas \cite{redesProfundas} \cite{redesProfundas2}. Aún así para este trabajo decidimos utilizar Modelos Ocultos de Markov mas modelos de Mezcla de Gaussianas (HMM+GMM) que, a partir de un corpus de datos de entrenamiento, extrae información acústica y genera un modelo probabilístico que permite sintetizar habla. Para esto utilizaremos HTS \cite{hts}, un framework de entrenamiento y síntesis de voz basado en HMM+GMM.


Consideramos que este método, si bien no nos permitirá obtener la mejor voz posible, es una tecnología madura y estable, con software libre disponible y de sencillo empleo, que nos permite realizar nuestros experimentos con relativa facilidad.


Como principal fuente de información utilizaremos la disertación doctoral \textit{``Simultaneous Modeling of phonetic and prosodic parameters, and characteristic conversion for hmm-based text-to-speech systems''} del Profesor Tadashi Kitamura, Nagoya Institute of Technology \cite{phoneticAndProsodic}, donde se describen de manera detallada las decisiones de diseño utilizadas en HTS, así como el modelado de parámetros para la construcción de HMMs y el modelado de cada fonema utilizando Mel-Cepstral.


En las siguiente secciones se detallarán las decisiones de metodología utilizadas a lo largo de la investigación, como así también detalles teóricos como el mapeo de fonemas necesario para adaptar el repertorio fonético del inglés al castellano, etc.

A modo de mapa conceptual, a continuación presentamos de manera esquematica los temas centrales que se abordan en este trabajo.
 
En la Sección \ref{dataPrepartion} se presenta las técnicas utilizadas para el etiquetado fonético de los distintos corpus sobre los que trabajaremos. 

Luego, en la Sección \ref{phoneMaping} se presenta un mapeo entre los fonos del castellano y los del inglés.

En la Sección \ref{modelInterpolation} y \ref{speakerAdaptativeTraining} presentamos las herramientas provistas por HTS para combinar modelos y poder sintetizar habla con distintos grados de mezcla fonética y prosódica de inglés-castellano.

En la Sección \ref{herramientas} se detallan aspectos técnicos del trabajo, así cómo especificaciones de cómo modela el audio HTS, estructuras utilizadas durante el entrenamiento y otras herramientas utilizadas. 

Por último, en la Sección \ref{evaluacionPerceptual}, intentamos validar que el sistema construido realmente cumple con las caracteristicas deseadas, exponiéndose los aspectos metodológicos de la evaluación y los resultados obtenidos.
