\noindent Un sistema de Text To Speech (TTS) es aquel que genera habla artificial a partir de un texto de entrada. En la actualidad estos sistemas se encuentran incluidos en diversas aplicaciones domesticas: navegación por GPS, asistentes personales inteligentes (como es el caso de SIRI), ayuda para personas no videntes, traducción automática, etc.

\noindent En las últimas décadas se han visto grandes progresos en este campo. sAlgunos sistemas son capaces de modelar con cierto grado de efectividad cuestiones tales como la el acento, el tono y la entonación de un hablante (es decir, su prosodia). Por otro lado tambien se ha visto un progreso en modelar emociones, o las intenciones discursivas (Cuestiones prosodicas que pueden darnos a entender que una oración es una pregunta, una orden, etc). 

\noindent Actualmente se considera que el estado del arte para la síntesis de voz es el entrenamiento con redes neuronales profundas (DNN). Aún así para este trabajo decidimos utilizar Modelos Ocultos de Markov mas modelos de Mezcla de Gaussianas (HMM+GMM) que, a partir de un corpus de datos de entrenamiento, extrae información acústica y genera un modelo probabilístico que permita sintetizar habla. Para esto utilizaremos HTS\cite{hts}, un framework de entrenamiento y sintesis de voz basado en HMM+GMM.

\noindent Consideramos que este método, si bien no nos permitirá obtener la mejor voz posible, nos permitirá entender mejor los features con los que trabajamos y por lo tanto modificarlos según consideremos conveniente.

\noindent En este trabajo de tesis se estudia una manera posible de generar un TTS basado en HMMs capaz de sintetizar habla en español con acento extranjero. Existen muchos motivos por los que podría construirse un sistema cone estas características. Por ejemplo en investigaciones de carácter lingüístico, podría querer utilizarse para vislumbrar limites en que un acento deja de parecernos local para pasar a sonar extranjero. Por otro lado en investigaciones de carácter psicológico, podría querer utilizarse para medir la confianza que deposita un oyente sobre hablantes de distinta nacionalidad: por poner un ejemplo, al pedir indicaciones de como llegar a algún lugar uno no deposita el mismo nivel de confianza si la persona que responde suena oriundo de la localidad que a alguien que suena extranjero. Por ultimo un sistema así podría quererse construir por temas puramente técnicos ya que permitiría generar una voz en castellano, por ejemplo, combinando una voz previamente construida en algún otro idioma y un pequeño corpus de datos en castellano.

\noindent Como principal fuente de información utilizaremos la disertación doctoral \textit{Simultaneous Modeling of phonetic and prosodic parameters, and characteristic conversion for hmm-based text-to-speech systems} del Profesor Tadashi Kitamura, Nagoya Institute of Technology \cite{phoneticAndProsodic}, donde se describe de manera detallada las desiciones de diseño utilizadas en HTS, así como el modelado de parámetros para la construcción de HMMs, el modelado de cada fonema utilizando Mel-Cepstral, etc.

\noindent Como introducción a este trabajo comenzaremos analizando las decisiones de metodología utilizadas a lo largo de la investigación, como así también detalles teóricos como el mapeo de fonemas necesario para adaptar el repertorio fonético del inglés al castellano, etc.

\noindent En las siguiente secciones se detallarán distintos temas que fueron necesarios abordar para llevar a cabo esta tesis. En orden de aparición estos son:

\begin{itemize}
\item Comenzando en la sección \ref{dataPrepartion} se presentan las técnicas utilizadas para el etiquetado fonético de los distintos corpus sobre los que trabajaremos.

\item En la sección \ref{phoneMaping} se presentará un mapeo entre los fonos del castellano y los del inglés.

\item En la sección \ref{modelInterpolation} y \ref{speakerAdaptativeTraining} presentamos las herramientas provistas por HTS para combinar modelos y poder sintetizar habla con distintos grados de mezcla fonética y prosódica de inglés-castellano.

\item En la sección \ref{herramientas} se detallaran aspectos técnicos del trabajo, así como especificaciones de como modela el audio HTS, estructuras utilizadas durante el entrenamiento y otras herramientas utilizadas. 

\item Por ultimo, en todo el capitulo \ref{evaluacionPerceptual}, intentaremos validar que el sistema construido realmente cumple con las caracteristicas deseadas, exponiendose en la sección \ref{interfaz} los aspectos metodologicos de la evaluación y en las subsiguientes secciones los resultados obtenidos.

\end{itemize}

%\noindent En el transcurso de este trabajo se espera además evaluar la prosodia y la fonética del modelo generado con estas características, como así también evaluar su inteligibilidad. 
