

Un sistema de Text To Speech (TTS) es aquel que genera habla artificial a partir de un texto de entrada. En la actualidad estos sistemas se encuentran incluidos en muchas aplicaciones domesticas, desde navegación por GPS, asistentes personales inteligentes (como es el caso de SIRI), ayuda para personas no videntes, traducción automática, etc.

En las últimas décadas se han visto grandes progresos en este campo, siendo capaces de modelar con cierto grado de efectividad cuestiones tales como la prosodia del hablante, emociones, etc. Una técnica bastante utilizada es la que utiliza modelos ocultos de markov (HMMs) que, a partir de un corpus de datos de entrenamiento, extrae información acústica y genera un modelo probabilístico que permita sintetizar habla. 

%Como desafío, estos corpus resultan dificultosos de obtener y segmentar y almacenar de manera correcta.

En la actualidad el campo de la síntesis utilizando HMMs presenta algunos interrogantes con respecto al entrenamiento y utilización de corpus de datos con hablantes de distintas lenguas [1,2]. 

En este trabajo de tesis se pretende presentar una manera posible de generar un TTS basado en HMMs capaz de sintetizar habla en español con acento extranjero. Las razones por las que podría querer diseñarse un sistema con estas características varían desde un punto de vista puramente técnico, ya que un sistema así permitiría la utilización de corpus de entrenamiento de hablantes no nativos para la generación de una nueva voz, hasta cuestiones lingüísticas, como es poder vislumbrar el limite en que un acento deja de parecernos local para pasar a ser extranjero.

En el transcurso de este trabajo se espera además evaluar la prosodia y la fonética del modelo generado con estas características, como así también evaluar su inteligibilidad. Además pretenderemos evaluar la efectividad de técnicas de speaker adaptation cuando se utilizan corpus de distintas nacionalidades con repertorios fonéticos muy disimiles (para este caso de estudio: castellano e ingles)

Para este trabajo nos basaremos fuertemente en la síntesis/análisis mel-cepstral, speech parameter modeling usando HMMs y speech parameter generation usando HMMs, como es descripto en la disertación doctoral Simultaneous Modeling of phonetic and prosodic parameters, and characteristic conversion for hmm-based text-to-speech systems del Professor Tadashi Kitamura del Nagoya Institute of Technology[3].

También se utilizarán las herramientas para la investigación y generación de nuevas voces Festival y Festvox, para el preprocesamiento de datos.