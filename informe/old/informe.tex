\documentclass[a4paper]{article}

\usepackage[spanish]{babel} % Le indicamos a LaTeX que vamos a escribir en español.
\usepackage[utf8]{inputenc} % Permite utilizar tildes y eñes normalmente
\usepackage{caratula} % Se puede descargar en ~> https://github.com/bcardiff/dc-tex
\usepackage[font={small,it}]{caption}
%hopefully no floting images
\usepackage{floatrow}
\floatplacement{figure}{H}
\floatplacement{table}{H}

\begin{document} % Todo lo que escribamos a partir de aca va a aparecer en el documento.
%fran
%\sloppy

% Completar los datos de la caratula
\titulo{Implementación y evaluación de un TTS de habla con distos grados de acento extranjero} 
\fecha{\today}
\materia{}
\grupo{}

% Completar los integrantes del grupo:)
\integrante{Negri, Franco }{893/13}{franconegri2004@hotmail.com}

\maketitle

\tableofcontents

\pagebreak

\section{Introducción}

\noindent Un sistema de Text To Speech (TTS) es aquel que genera habla artificial a partir de un texto de entrada. En la actualidad estos sistemas se encuentran incluidos en muchas aplicaciones domesticas, como por ejemplo navegación por GPS, asistentes personales inteligentes (como es el caso de SIRI), ayuda para personas no videntes, traducción automática, etc.

\noindent En las últimas décadas se han visto grandes progresos en este campo, siendo capaces de modelar con cierto grado de efectividad cuestiones tales como la prosodia del hablante, sus emociones, etc. Si bien actualmente se considera que el estado del arte para la síntesis es el entrenamiento con redes neuronales profundas (DNN), una técnica todavía utilizada es la que utiliza Modelos Ocultos de Markov mas modelos de Mezcla de Gaussianas (HMM+GMM) que, a partir de un corpus de datos de entrenamiento, extrae información acústica y genera un modelo probabilístico que permita sintetizar habla. 

\noindent Consideramos que este método, si bien no nos permitirá obtener la mejor voz posible, nos permitirá entender mejor los features con los que trabajamos y por lo tanto modificarlos según consideremos conveniente, algo que puede volverse mas complicado cuando se esta trabajando con redes neuronales profundas.

\noindent En este trabajo de tesis se estudia una manera posible de generar un TTS basado en HMMs capaz de sintetizar habla en español con acento extranjero. Las razones por las que podría querer diseñarse un sistema con estas características varían \expandir desde un punto de vista puramente técnico, ya que un sistema así permitiría la utilización de corpus de entrenamiento de hablantes no nativos para la generación de una nueva voz, pasando por cuestiones lingüísticas, como es poder vislumbrar el limite en que un acento deja de parecernos local para pasar a ser extranjero y cuestiones psicológicas: lograr distintos efectos sobre el usuario, quien podría reaccionar distinto ante diferentes acentos.

posibles temas de psicolgia del hablante, expandir

\calzarMasAdelante
%Además pretenderemos evaluar la efectividad de técnicas de speaker adaptation cuando se utilizan corpus de distintas nacionalidades con repertorios fonéticos disimiles (para este caso de estudio: castellano e inglés)

\reescribir
\noindent Como principal fuente de información utilizaremos la disertación doctoral \textit{Simultaneous Modeling of phonetic and prosodic parameters, and characteristic conversion for hmm-based text-to-speech systems} del Profesor Tadashi Kitamura, Nagoya Institute of Technology\cite{phoneticAndProsodic}, donde se describe de manera detallada las desiciones de diseño utilizadas en HTS, así como el modelado de parámetros para la construcción de HMMs, el modelado de cada fonema utilizando Mel-Cepstral, etc.

\noindent Como introducción a este trabajo comenzaremos analizando las decisiones de metodología utilizadas a lo largo de la investigación, como así también detalles teóricos como el mapeo de fonemas necesario para adaptar el repertorio fonético del inglés al castellano, etc.

\noindent En las siguiente secciones se detallarán distintos temas que fueron necesarios abordar para llevar a cabo esta tesis. En orden de aparición estos son:

\begin{enumerate}
\item Realizar un etiquetado fonético de distintos corpus de audios.

\item Realizar un mapeo entre los fonos del castellano y los del inglés. Estos serán necesarios en el paso siguiente donde será requerimiento indispensable tener modelados los mismos fonos para todos los modelos.

\item Realizar el entrenamiento de los HMM+GMM. Para esto contaremos con el framework de modelado de HMMs HTS. 

\item Utilizar las herramientas provistas por HTS para interpolar entre modelos y poder sintetizar habla con distintos grados de fonética y prosodia inglesa.

\item Por último se detallaran aspectos técnicos del trabajo, así como especificaciones de como modela el sonido HTS, Festival y Festvox etc. 

\end{enumerate}

\noindent En el transcurso de este trabajo se espera además evaluar la prosodia y la fonética del modelo generado con estas características, como así también evaluar su inteligibilidad. 

\pagebreak
\section{Metodología}
\section{Objetivo}

El objetivo de este trabajo se sentrará en explorar diversas tecnicas para sintetizar habla en español con acento extranjero. 


Para ello utilizaremos HTS para el entrenamiento y sintesis del habla y Festival y Festvox para realizar el etiquetado automatico de los datos.


Partiendo de tres corpus de datos, dos de ellos en castellano y uno en ingles, generaremos tres HMMs distintos (dos HMMs en castellano y uno en ingles) y utilizando herramientas provistas por HTS interpolaremos entre ellos para obtener distintos grados de acento ingles a la hora de sintetizar.


Dado que el castellano y el ingles no utilizan los mismos simbolos foneticos, una problematica a resolver será mapear fonemas del ingles con su fonema mas similar del cercano.

\section{Metodología}

\subsection{Preparación De los datos}

Como ya adelantamos, en este trabajo contamos con tres corpus de datos dispoinbles:

\begin{itemize}
\item secyt-mujer: 741 oraciones, $48$ minutos de habla.
\item loc1_pal: 1593 oraciones, $2$ horas y $26$ minutos de habla.
\item CMU-ARCTIC-SLT: 1132 oraciones, 56 minutos de habla.
\end{itemize}

Dadas la cantidad de horas de audio disponibles, tanto para loc1_pal como para CMU-ARCTIC-SLT decidimos utilizar alineamiento forzado para obtener las transcripciones foneticas necesarias para el entrenamiento. Para esto se utilizó Festival y Festvox que a partir de los audios y sus transcripciones grafemicas, permite realizar EHMM alignment sobre el corpus de datos. Para secyt-mujer contabamos previamente con las transcripciones foneticas ya realizadas por lo que decidimos utilizar estas. 


Por otra parte, festival nos permitirá generar features contextuales sobre cada fonema, como el fonema que lo precede, cantidad de palabras en la oración, si la silaba en la que se encuentra esta acentuada, etc. Mas adelante en este trabajo se explicará de que manera son utilizados estos features.


Para este trabajo todos los audios usarán sampling rate de 48kHz, precisión de 16 bits, mono.


Cosas para hablar:
contextual factors: cuales son, para que sirven.
desarrollar generacion de uternaces: secty alineaminento mixto: tiempos a mano, features automaticos.
el etiquetado de cmu_arctic es en ingles y mapeando al castellano.

\subsection{Entrenamiento Con HTS}

Para este trabajo se utilizará HTS para el entrenamiento de los diversos modelos foneticos.

Cosas para hablar:
5 fonemas.
hmms
arboles de decición.

\subsection{Sintesis utilizando hts_engine}

Luego del entrenamiento y con los modelos foneticos ya generados utilizaremos hts_engine para interpolar entre ellos y lograr nuevos modelos que mezclen los features acusticos con distintos grados de ingles y de castellano.

Un desafío que se presenta para este trabajo es el mapeo de los fonemas del ingles al castellano. Para empezar, la transcripcion fonetica realizada por festival de las oraciones en ingles puede utilizar 50 simbolos distintos, mientras que la transcripción fonetica del castellano utiliza 31. Habiendo ademas muchos simbolos sin equivalencia. (por ejemplo, con el fonema /rr).

Para resolver esto desarollamos una solución adhoc que consist en desarrollar una función sobreyectiva que permita tener cubiertos los 31 fonemas del castellano por alguno del ingles.

Cosas para hablar:
mapeo utilizado mostrar.
Expandir en que consiste la interpolación.
\pagebreak
\section{Experimentación}

En la siguiente aparatado intentamos validar dos hipótesis: que el modelo generado realmente puede ser identificado como un hablante extranjero de habla inglesa y al mismo tiempo que este posee un grado de inteligibilidad aceptable.

Para eso se condujo una encuesta perceptual donde a cada participante se le presentó una oración sintetizada con distintos grados de mezcla de español e ingles y se le pidió que la transcribiera y que intentara identificar la nacionalidad del hablante. Para evitar que el participante pudiera deducir las palabras a partir de las palabras vecinas, las mismas son generadas de manera semánticamente impredecible. Esto significa que a partir de una lista de sustantivos, adjetivos, determinantes y verbos se generan oraciones de manera aleatoria con la estructura:

$$\textnormal{\textit{Determinante Adjetivo Sustantivo Verbo Determinante Sustantivo}}$$

Luego, para asegurarnos de estar cubriendo todos los posibles fonos del castellano, las oraciones son modificadas para ser fonéticamente balanceadas. Esto querrá decir que incluimos entre cinco y diez veces cada fono perteneciente a una consonante (presente en el repertorio del castellano) y al menos veinte veces cada fono perteneciente a una vocal.

Los oraciones finalmente generadas fueron:

\begin{itemize}
\item Oración 1: Mi montaña aguileña recorrió la esquina
\item Oración 2: Aquel fuerte vidrio prefirió aquel botón
\item Oración 3: Este enjoyado juez comprará nuestro corchete
\item Oración 4: Tu estrecho posavasos gritó la fechoría
\item Oración 5: Nuestro nublado tigre concluyó a este chupetín
\item Oración 6: Su profundo riñón apoyó a Julio
\item Oración 7: El frío churrasco oyó lo de Polonia
\item Oración 8: Las acongojadas cotorras sonrieron a mi círculo
\item Oración 9: Ese gruñón perro prometió a esos cuñados
\item Oración 10: El nudillo Argentino perdió su vaso
\end{itemize}

Para cada uno de estos diez oraciones se varió el nivel de mezcla entre $30\%$ de ingles, $70\%$ castellano hasta $70\%$ de ingles, $30\%$ de castellano, $10\%$ cada vez. De esta manera, para cada oración habrá 5 mezclas diferentes, lo que hace un total de $50$ audios sintetizados diferentes.

La encuesta se realiza a través de internet, con el mismo set de instrucciones para todos los participantes y pidiendo como requerimiento la utilización de auriculares. Cada participante podía contestar un máximo $5$ veces (otorgándoles siempre audios distintos).

El objetivo de la misma es conseguir para cada uno de los $50$ audios sintetizados, $5$ respuestas, momento en el cual se cierra la posibilidad de contestar.

La misma se lleva a cavo desde el $18$ de octubre de $2017$ hasta el primero de diciembre del mismo año, tiempo durante el cual fue publicada en distintas redes sociales y listas de emails de la facultad.

Con el objetivo de no influir en las respuestas de los participantes, se procuró darles la información mínima indispensable para completar la encuesta. Por este motivo, en ningún momento de la encuesta se especifica el objetivo del estudio.

Con la intención de estandarizar los resultados, fue requisito obligatorio utilizar auriculares para la encuesta. También se le pidió a cada participante que la realizara en un lugar silencioso y tranquilo.

\section{Interfaz}

En este apartado se presenta la interfaz utilizada para realizar la encuesta junto con las decisiones de diseño más relevantes. 

En la figura \ref{personalData} se presenta la pagina principal en la que todos los participantes fueron recibidos.

\begin{figure}[htp]
\begin{center}
\fbox{\includegraphics[scale=0.6]{estudio_online/estudio1.png}}
\end{center}
\caption{Datos Personales}
\label{personalData}
\end{figure}

A fin de conocer de manera general la demografía encuestada, a cada participante se le pidió que indique el rango correspondiente a su edad, yendo desde $18$ a $25$, $26$ a $35$, y así de diez en diez.

Se les pidió, además, que indicara su genero: masculino, femenino, otro, no contesta y la provincia donde pasó sus primeros diez años de vida. Consideramos que estos datos son importantes para el estudio ya que dependiendo de ellos los resultados variarán indefectiblemente, la transcripción que obtendremos de un participante de $50$ años de capital federal será distinta a la de alguien de $18$ años de Córdoba. El diferente uso de los alofonos, modismos y variantes prosódicas y capacidades auditivas jugarán un papel importante en la interpretación de la oración y la apreciación del origen del hablante.

Una vez que completados estos datos, se le presenta otra vista con las instrucciones especificas para completar la encuesta (figura \ref{instrucciones}).

\begin{figure}
\begin{center}
\fbox{\includegraphics[scale=0.6]{estudio_online/estudio2.png}}
\end{center}
\caption{Instrucciones}
\label{instrucciones}
\end{figure}

Una vez presionado el botón de ``entendido!'' se les presentaba un audio, que podían escuchar un máximo de $2$ veces, una caja de texto libre donde plasmar la transcripción del mismo y una caja de texto libre donde podían escribir la nacionalidad correspondiente a la voz (figura \ref{transcripcion})

\begin{figure}
\begin{center}
\fbox{\includegraphics[scale=0.6]{estudio_online/estudio3.png}}
\end{center}
\caption{Transcripción}
\label{transcripcion}
\end{figure}

Una vez que la respuesta es guardada, se le preguntaba si quiere continuar transcribiendo otro audio, o de caso de haber contestado cinco veces, se le presentaba un mensaje donde se le indicaba que ya podía cerrar la encuesta (figura \ref{continuar}).

\begin{figure}
\begin{center}
\fbox{\includegraphics[scale=0.6]{estudio_online/estudio4.png}}
\end{center}
\caption{Dialogo Final}
\label{continuar}
\end{figure}
\pagebreak
\section{Trabajo Futuro}
\subsection{Trabajo Futuro}

Como fue discutido en la sección de experimentación, una interpolación general puede producir que ciertos fonemas se alejen demasiado del fonema real del castellano, disminuyendo la inteligibilidad de la voz sintetizada. Un posible camino a seguir es realizar una interpolación controlada que permita regular cada fonema por separado. Para fonemas que puedan resultar problematicos como el caso de la /r bibrante el grado de interpolación podría dejarse mas cercano al castellano, mientras que para fonemas con comportamientos mas similares el grado de interpolación podría llevarse mas cerca del modelo ingles.

%expandir
\pagebreak
\section{Apendice}
\section{Apendice}
\subsection{Lista de Fonemas}
\begin{center}
 \begin{tabular}{ | l | l |}
 \hline
Castellano & Ingles \\ \hline
a & aa \\ \hline
a1 & ae \\ \hline
b & ah \\ \hline
ch & ao \\ \hline
d & aw \\ \hline
e & ax \\ \hline
e1 & ay \\ \hline
f & b \\ \hline
g & ch \\ \hline
i & d \\ \hline
i0 & s \\ \hline
i1 & dh \\ \hline
k & eh \\ \hline
l & er \\ \hline
ll & ey \\ \hline
m & f \\ \hline
n & g \\ \hline
ny & hh \\ \hline
o & ih \\ \hline
o1 & iy \\ \hline
p & jh \\ \hline
r & k \\ \hline
rr & l \\ \hline
s & m \\ \hline 
t & n \\ \hline
u & ng \\ \hline
u0 & ow \\ \hline 
u1 & oy \\ \hline
x & p \\ \hline
- & r \\ \hline
- & sh \\ \hline
- &t \\ \hline
- &th \\ \hline
- &uh \\ \hline
- &uw \\ \hline
- &v \\ \hline
- &w \\ \hline
- &y \\ \hline
- &z \\ \hline
- &zh \\
    \hline
    \end{tabular}
\end{center}

\subsection{Mapeo Fonemas del Ingles-Castellano}

\begin{center}
 \begin{tabular}{ | l | l |}
 \hline
Ingles  & Castellano \\ \hline
aa & a1\\ \hline 
ae & a\\ \hline 
ao & o\\ \hline 
ou & o1\\ \hline 
b & b\\ \hline 
ch & ch\\ \hline 
d & d\\ \hline 
dh & d\\ \hline 
dx & dx\\ \hline 
eh & e\\ \hline 
el & e1\\ \hline 
em & em\\ \hline 
en & en\\ \hline 
er & er\\ \hline 
ei & ei\\ \hline 
f & f\\ \hline 
g & notUsed3\\ \hline 
hh & h\\ \hline 
hv & hv\\ \hline 
ih & i1\\ \hline 
iy & i\\ \hline 
k & k\\ \hline 
l & l\\ \hline 
m & m\\ \hline 
n & n\\ \hline 
nx & n\\ \hline 
ng & ng\\ \hline 
p & p\\ \hline 
r & r\\ \hline 
s & s\\ \hline 
t & t\\ \hline 
uh & u1\\ \hline 
uw & u\\ \hline 
w & u0\\ \hline 
th & notUsed\\ \hline 
v & v\\ \hline 
jh & ll\\ \hline 
y & y\\ \hline 
sh & sh\\ \hline 
zh & zh\\ \hline 
z  & notUsed2\\ \hline 
\end{tabular}
\end{center}

\subsection{Parametros utilizados para el entrenamiento}


\pagebreak
\section{Referencias}
\begin{thebibliography}{9}
\bibitem{redesProfundas} S. O. Arık, M. Chrzanowski, A. Coates, G. Diamos, A. Gibiansky, Y. Kang, X. Li, J. Miller, J. Raiman,S. Sengupta, and M. Shoeybi. Deep Voice: Real-time neural text-to-speech. In ICML, 2017a.
\bibitem{redesProfundas2} Arik, Sercan \& Diamos, Gregory \& Gibiansky, Andrew \& Miller, John \& Peng, Kainan \& Ping, Wei \& Raiman, Jonathan \& Zhou, Yanqi. (2017). Deep Voice 2: Multi-Speaker Neural Text-to-Speech. 
\bibitem{prosodiaYEntonacion} Yamagishi, Junichi \& ONISHI, Koji \& Masuko, Takashi \& Kobayashi, Takao. (2005). Acoustic Modeling of Speaking Styles and Emotional Expressions in HMM-Based Speech Synthesis. IEICE Transactions on Information and Systems. E88D. 10.1093/ietisy/e88-d.3.502. 
\bibitem{prosodiaYEntonacion2} Nose, Takashi \& Yamagishi, Junichi \& Masuko, Takashi \& Kobayashi, Takao. (2007). A Style Control Technique for HMM-Based Expressive Speech Synthesis. IEICE Transactions. 90-D. 1406-1413. 10.1093/ietisy/e90-d.9.1406. 
\bibitem{praat} Boersma, Paul (2001). Praat, a system for doing phonetics by computer. Glot International 5:9/10, 341-345.
\bibitem{hts} http://hts.sp.nitech.ac.jp/?Download
\bibitem{secytMujer} Automatic determination of phrase breaks for Argentine Spanish, Humberto M. Torres \& Jorge A. Gurlekian, Laboratorio de Investigaciones Sensoriales CONICET, University of Buenos Aires, Argentina
\bibitem{loc1pal} Torres, Humberto \& Gurlekian, Jorge \& Cossio-Mercado, Christian. (2012). Aromo: Argentine Spanish TTS System.
\bibitem{cmuArtic} Kominek, John \& W Black, Alan. (2004). The CMU Arctic speech databases. SSW5-2004.
\bibitem{statisticalParam} Statistical Parametric Speech Synthesis Based on Speaker and Language Factorization, Heiga Zen, Member, IEEE, Norbert Braunschweiler, Sabine Buchholz, Mark J. F. Gales, Fellow, IEEE, Kate Knill, Member, IEEE, Sacha Krstulovic, and Javier Latorre, Member, IEEE
\bibitem{whyItSucked} Simultaneous Modeling of phonetic and prosodic parameters, and characteristic conversion for hmm-based text-to-speech systems, takayoshi yoshimura, pag. 28
\bibitem{speakerAdaptativeTrainingLink} http://hts.sp.nitech.ac.jp/archives/2.3/HTS-demo\_CMU-ARCTIC-ADAPT.tar.bz2
\bibitem{speakerSim} Speaker Similarity Evaluation Of Foreign-accented Speech Synthesis Using Hmm-based Speaker Adaptation, Mirjam Wester, Reima Karhila
\bibitem{phoneticAndProsodic} Simultaneous Modeling of phonetic and prosodic parameters, and characteristic conversion for hmm-based text-to-speech systems, takayoshi yoshimura
\bibitem{phoneticCapturing} SUB-PHONETIC MODELING FOR CAPTURING PRONUNCIATION VARIATIONS FOR CONVERSATIONAL SPEECH SYNTHESIS, Kishore Prahallad, Alan W Black and Ravishankhar Mosur https://www.cs.cmu.edu/~awb/papers/ICASSP2006/0100853.pdf
% como funciona el ehmm alignment
\bibitem{spanishInfluencedEnglish} Transcription of Spanish and Spanish-Influenced English, Brian Goldstein
\bibitem{SpekerInterpolationRef} Speaker Interpolation in HMM-based Speech Synthesis System, Takayoshi Yoshimura1, Takashi Masuko2, Keiichi Tokuda1, Takao Kobayashi2 and Tadashi Kitamura1
\bibitem{festivalDownload} http://www.cstr.ed.ac.uk/downloads/festival/2.4/
\bibitem{festvoxDownload} http://festvox.org/download.html
\bibitem{speechToolDownload} http://www.cstr.ed.ac.uk/projects/speech\_tools
\end{thebibliography}


\end{document}
