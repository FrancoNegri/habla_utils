\subsection{Experimentación}

En la siguiente aparatado intentaremos validar dos hipótesis: que el modelo generado realmente puede ser identificado como un hablante extranjero y al mismo tiempo que este posee un grado de inteligibilidad aceptable.

Para eso se condujo una encuesta perceptual donde, dado un participante, se le presentó un audio con una oración semánticamente impredecible, fonéticamente balanceada y con distintos grados de mezcla de español e ingles, se le pidió que la transcribiera y que intentara identificar la nacionalidad del mismo.

Para la experimentación, se generaron diez oraciones distintas variando el nivel de mezcla de los modelos generados entre $30\%$ de ingles - $70\%$ castellano hasta $70\%$ de ingles - $30\%$ de castellano.

La encuesta se realizó a través de internet, con el mismo set de instrucciones para todos los participantes y pidiendo como requerimiento la utilización de auriculares. Cada participante podía contestar como máximo 5 veces a la encuesta (otorgandoseles audios siempre distintos).
