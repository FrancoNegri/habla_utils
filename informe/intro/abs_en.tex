%\begin{center}
%\large \bf \runtitle
%\end{center}
%\vspace{1cm}
\chapter*{\runtitle}

El presente trabajo tiene como objetivo generar un sistema de síntesis de habla en castellano con acento inglés. Si bien actualmente los métodos mas avanzados para generar habla utilizan técnicas de aprendizaje automático basados en redes neuronales, en este trabajo utilizamos técnicas basadas en HMM+GMM por su practicidad. Utilizando esta técnica construimos modelos en castellano y en inglés y luego, interpolando sus características acústicas, intentamos producir un modelo capaz de sintetizar habla que resulte inteligible pero que contenga caracteristicas atribuibles a un extranjero angloparlante hablando castellano. Para este trabajo la interpolación de características acústicas ocurre a nivel de fono, por lo que, para realizar la mezcla de modelos es necesario compatibilizar los fonos de ambos idiomas. Por este motivo, parte del trabajo consiste en desarrollar un mapeo entre los repertorios fonéticos del inglés y el castellano que resulte natural. Por último evaluamos el nivel de efectividad de nuestro método de manera experimental. Para ello desarrollamos una metodología de evaluación que nos permita medir tanto la ininteligibilidad de un audio, como el origen atribuido por los participantes.

\bigskip

\noindent\textbf{Keywords:} Síntesis de habla, HMM, HTS, GMM, acento extranjero, aprendizaje automático.