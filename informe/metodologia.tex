\section{Objetivo}

El objetivo de este trabajo se sentrará en explorar diversas tecnicas para sintetizar habla en español con acento extranjero. 


Para ello utilizaremos HTS para el entrenamiento y sintesis del habla y Festival y Festvox para realizar el etiquetado automatico de los datos.


Partiendo de tres corpus de datos, dos de ellos en castellano y uno en ingles, generaremos tres HMMs distintos (dos HMMs en castellano y uno en ingles) y utilizando herramientas provistas por HTS interpolaremos entre ellos para obtener distintos grados de acento ingles a la hora de sintetizar.


Dado que el castellano y el ingles no utilizan los mismos simbolos foneticos, una problematica a resolver será mapear fonemas del ingles con su fonema mas similar del cercano.

\section{Metodología}

\subsection{Preparación De los datos}

Como ya adelantamos, en este trabajo contamos con tres corpus de datos dispoinbles:

\begin{itemize}
\item secyt-mujer: 741 oraciones, $48$ minutos de habla.
\item loc1_pal: 1593 oraciones, $2$ horas y $26$ minutos de habla.
\item CMU-ARCTIC-SLT: 1132 oraciones, 56 minutos de habla.
\end{itemize}

Dadas la cantidad de horas de audio disponibles, tanto para loc1_pal como para CMU-ARCTIC-SLT decidimos utilizar alineamiento forzado para obtener las transcripciones foneticas necesarias para el entrenamiento. Para esto se utilizó Festival y Festvox que a partir de los audios y sus transcripciones grafemicas, permite realizar EHMM alignment sobre el corpus de datos. Para secyt-mujer contabamos previamente con las transcripciones foneticas ya realizadas por lo que decidimos utilizar estas. 


Por otra parte, festival nos permitirá generar features contextuales sobre cada fonema, como el fonema que lo precede, cantidad de palabras en la oración, si la silaba en la que se encuentra esta acentuada, etc. Mas adelante en este trabajo se explicará de que manera son utilizados estos features.


Para este trabajo todos los audios usarán sampling rate de 48kHz, precisión de 16 bits, mono.


Cosas para hablar:
contextual factors: cuales son, para que sirven.
desarrollar generacion de uternaces: secty alineaminento mixto: tiempos a mano, features automaticos.
el etiquetado de cmu_arctic es en ingles y mapeando al castellano.

\subsection{Entrenamiento Con HTS}

Para este trabajo se utilizará HTS para el entrenamiento de los diversos modelos foneticos.

Cosas para hablar:
5 fonemas.
hmms
arboles de decición.

\subsection{Sintesis utilizando hts_engine}

Luego del entrenamiento y con los modelos foneticos ya generados utilizaremos hts_engine para interpolar entre ellos y lograr nuevos modelos que mezclen los features acusticos con distintos grados de ingles y de castellano.

Un desafío que se presenta para este trabajo es el mapeo de los fonemas del ingles al castellano. Para empezar, la transcripcion fonetica realizada por festival de las oraciones en ingles puede utilizar 50 simbolos distintos, mientras que la transcripción fonetica del castellano utiliza 31. Habiendo ademas muchos simbolos sin equivalencia. (por ejemplo, con el fonema /rr).

Para resolver esto desarollamos una solución adhoc que consist en desarrollar una función sobreyectiva que permita tener cubiertos los 31 fonemas del castellano por alguno del ingles.

Cosas para hablar:
mapeo utilizado mostrar.
Expandir en que consiste la interpolación.